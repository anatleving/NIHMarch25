\section{Innovation}

\boldstart{Skull clearing:}
Optical imaging techniques have significantly advanced our understanding of the brain structure and functionality. However, the brain is mostly inaccessible as light is heavily scattered by the skull. Almost all brain research starts with invasive skull opening steps. Here we plan to employ novel alternatives which use chemical agent to clear the skull and make it transparent so optical instruments can see through it~\cite{Li2022TIS}. 
\Anat{Hillel please write here a paragraph on the advantages of skull clearing}

\boldstart{Wavefront shaping:} Even after skull opening or clearing, scattering remains one of the most significant barriers in biomedical research, making it impossible to see beyond superficial tissue layers. Many creative imaging strategies have been developed over the years, but most are not suitable for light-starved fluorescent imaging challenges. Techniques such as confocal microscopy~\cite{ConfocalMicroscopyOverView2020} and optical coherence tomography (OCT)~\cite{OCTOverview2016} achieve deeper imaging by filtering out scattered photons and isolating ballistic ones. Despite this progress, these techniques are inherently limited to thin layers as the number of ballistic photons decays rapidly within the scattering material.

Another approach for deep tissue imaging involves computational corrections~\cite{Metzler23NeuWS,YeminyKatz2021,haim2023imageguidedcomputationalholographicwavefront,Balondrade_2024,Kang2017,Najar2024, Zhu22, Baek_2023,Jeong2018,Gil2024,Yonghyeon2022}, or diffraction tomography approaches~\cite{Kim2013,Horstmeyer:16,Chowdhury:17,Chowdhury:19,Chen:20,Zhou:20,ChoiLyers2023,liu2022recoverycontinuous3drefractive,Xue:22,he2024fluorescencediffractiontomographyusing,Kamilov:15,Sun:18,PMID:17694065,Choi2011,Choi2015,Badoneaay7170,Kwon2023,ChoiLyers2023,zhang2024deepimaginginsidescattering,Choi2023AngWavelength}, which attempt to measure the full set of wavefronts scattered through a tissue volume and use them for a digital reconstruction of the 3D refractive index variations in the tissue. However, fluorescent data is often too weak and noisy to support such digital correction.

Wavefront shaping stands out as a promising alternative because the correction is done within the optical path, before digitization. Unlike ballistic filtering approaches, wavefront shaping utilizes {\em all light photons}.
 
 
 There have been multiple previous attempts to use spatial light modulators for aberration correction, yet none have been practically applied for aberration correction deep inside scattering tissue using weak biological data. One class of techniques, known as adaptive optics~\cite{Booth2014,Ji2017review,HampsonBooth21review,wang2015direct}, has been successfully applied to brain imaging~\cite{ji2009adaptive,ji2012characterization,Wang2014Multiplexed,Papadopoulos16,Rodriguez2021Adaptive}. Originally, adaptive optics aimed to correct aberrations in the optical path, or those formed by the mismatch between the refractive index of the tissue and that of the optics. Such aberrations are of low order and occur relatively far from the target of interest (i.e., in the optical path and not inside the tissue). As a result, one modulation can correct a large field of view. However, adaptive optics approaches are not designed to correct tissue scattering. First, because tissue scattering inherently occurs in a 3D volume, the aberration of every point in the volume is different. Second, tissue aberration involves complicated speckle patterns that cannot be well approximated by simple low-order corrections.
 
   
 Wavefront shaping techniques have advanced adaptive optics ideas and demonstrated the possibility of correcting not only optical aberrations but also those produced by thick, highly-scattering samples~\cite{Vellekoop:07,Yaqoob2008,Vellekoop2010,Vellekoop2012}. For a review of wavefront shaping progress, see~\cite{Horstmeyer15,YU2015632,Gigan22}. Wavefront shaping holds significant potential for revolutionizing biomedical imaging, enabling very deep tissue imaging with high signal-to-noise ratios (SNR). Despite many recent algorithmic developments~\cite{Horstmeyer15,Tang2012,Katz:14,Wang20142PAdaptive,Liu2018,Fiolka:12,Jang:13,Xu11,Wang2012,Kong:11,Vellekoop2012,YeminyKatz2021,Stern:19,Daniel:19,Boniface:19,Dror22}, wavefront shaping has not been widely applied to real biological data until recently, due to two main challenges. First, correcting such severe aberrations requires a very large number of degrees of freedom (aberration modes), often involving  megabytes of SLM pixels. Estimating the desired modulation using non-invasive, guide-star-free feedback is challenging given the low SNR of realistic fluorescent sources. Second, since scattering results from the 3D brain structure, any wavefront shaping modulation can only correct extremely localized regions of the tissue volume, making it unclear how to use such systems for wide field of view imaging.
 
 
 We propose the first in-vivo application of wavefront shaping to correct scattering aberration in the mouse brain. To make this practical, we rely on two key properties. First, the range of aberrations we aim to tackle is more challenging than those addressed by adaptive optics approaches but is significantly more modest compared to the classical definition of wavefront shaping. For the brain depths we target, a typical speckle pattern resulting from a diffraction-limited source inside the volume has a diameter of a few dozen pixels, and thus an area of a few hundred pixels. This bounds the number of modes we need to correct. Second, since we aim to image a sparse neural population, {\em we do not need a global wide field of view correction; instead, we can use different SLM pixels to correct different neurons.}
 
 Our research builds on recent innovations achieved by PI Levin's lab~\cite{DrorNatureComm24}, who demonstrated, for the first time, non-invasive imaging of weak ex-vivo fluorescent neurons using affordable 1P excitation. Their system is carefully designed to operate under the realistic low SNR values of fluorescent neural emission.
  
Overall, the proposed project offers a novel approach to deep tissue imaging. If successful, it has the potential to revolutionize brain imaging, enabling us to image significantly deeper into the brain with much higher signal-to-noise ratios.
 
 
% 
% This project aims to utilize these innovative techniques for in-vivo mice brain imaging.
%  
% However, until recently it was not applied to real biological data, due to  the difficulty in estimating the modulation using guide-star free, non-invasive feedback. Additionally, early wavefront shaping systems required high SNRs that realistic biological fluorescent sources could not provide. These obstacles were recently overcome by PI Levin's lab, who demonstrated, for the first time,  non-invasive imaging of weak ex-vivo fluorescent neurons using affordable 1P excitation. This project aims to utilize these innovative techniques for in-vivo mice brain imaging.
%
%Another difficulty which hindered the practical applicability of wavefront shaping system is the fact that since scattering results from the 3D brain structure, any wavefront shaping modulation can correct extremely localized regions of the tissue volume, and it is not clear how to use such systems for wide field of view imaging. We argue that this is not a problem when aiming to image a sparse neural population, as we can simply use different SLM pixels to correct different neurons.  Below we analyze the 
%
%The wavefront shaping system proposed in this project forms an high-risk high-gain research. 




\subsection{Team}
This project will be led by Prof. Laura Waller from the Berkeley EECS department, in collaboration with Prof. Hillel Adesnik from the Berkeley Neurobiology department. Prof. Anat Levin from the department of ECE at the Technion, Israel will join as a sub-contractor. Profs. Waller and Adesnik have a long history of successful collaboration, and Prof. Levin has been a frequent visitor.
% and is planning a year-long sabbatical at the Waller lab starting in the summer of 2025.
While forming a foreign component, Prof. Levin is the worldwide expert in wavefront shaping and the first one who managed to apply this concept on real biological data. Her expertise are necessary to the project success and such acknowledge is not currently available inside the US.    