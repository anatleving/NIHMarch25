\section{Significance}
\Anat{The following text contains a lot of references from the AdesnikWaller 2003 NIH proposal. I did not check all of them and it is possible that some are out of context}

Advances in neural imaging and perturbation have transformed our understanding of the neural
codes of sensation, cognition, and behavior. In particular, optical methods that leverage genetically encoded
sensors and actuators of neural activity have gained widespread adoption because they provide direct access
to genetically defined cell types, and they permit dense sampling and perturbation of neural circuits with very
high resolution. Approaches that combine  calcium imaging with  optogenetics~\cite{packer2014simultaneous,mardinly2018precise,yang2018simultaneous,rickgauer2014simultaneous,paluch2015all}
have yielded key new insights into neural coding because they can casually relate features of neural activity to network dynamics and behavior~\cite{marshel2019cortical,carrillo2019controlling,robinson2020targeted,daie2021targeted,carrillo2016imprinting,gill2020precise,forli2018two}. However, since calcium imaging has low temporal
resolution~\cite{chen2013ultrasensitive}, only indirectly measures spiking activity, and provides no access to subthreshold voltage
responses, these systems cannot probe fundamental features of the neural code that occur on faster timescales~\cite{gollisch2008rapid,butts2007temporal,berry1997structure,diesmann1999stable} or are contained in subthreshold potentials~\cite{sachidhanandam2013membrane,adesnik2017synaptic}. To overcome this  gap, newly developed voltage sensors have gained increased interest.
Voltage imaging
can directly report supra- and sub-threshold activity with millisecond resolution from specific cell types. 
Measuring the voltage of large numbers of
neurons with millisecond precision and cellular resolution in behaving animals  will enable neuroscientists to obtain major new insights into brain function.


Despite significant advancements, brain imaging methods still face substantial challenges due to scattering. The first major obstacle is the skull, a highly scattering layer that blocks light penetration. Most brain research today begins with invasive craniotomy operations. Even after opening the skull, scattering severely limits the effectiveness of imaging techniques to the superficial layers of the brain. To access deeper brain layers, researchers often resort to invasive approaches, using devices such as cannula~\cite{Mohammed2016Integrative}, gradient-index lenses~\cite{Jung2004InVivo,jennings2019interacting}, or microprisms~\cite{Andermann2013Chronic}. Without such waveguides, brain imaging using single-photon (1P) fluorescent excitation is currently limited to the first 200$\mu m$ of brain tissue. Expensive two-photon (2P) excitation lasers, which operate at longer wavelengths, can penetrate up to 700$\mu m$ before scattering becomes problematic, but they still cannot reach very deep brain layers. Moreover, while the longer wavelengths used by 2P lasers suffer less from scattering, the emitted light has a shorter wavelength and is thus severely scattered. To address this, 2P imaging systems use fast scanning systems~\cite{wu2020kilohertz}\Anat{more refs for scanning here?} that sequentially excite one point in the volume at a time and collect all the scattered photons with a single unit PMT detector. These scanning techniques are suitable for measuring calcium activity. However, with the growing interest in voltage imaging and the need to measure  activity at millisecond resolution over a large field of view, achieving sufficiently fast scanning remains challenging.


Instead of scanning the entire brain volume, it is more advantageous to monitor neural activity in specific, spatially distributed, user-defined groups of neurons. For instance, one could target a select group based on their responsiveness to a particular sensory stimulus or during a specific moment in a behavioral task. In many brain regions, only a sparse, distributed subset of neurons is active under any given condition~\cite{poo2009odor,diamantaki2016sparse,hromadka2008sparse,crochet2011synaptic,peron2015cellular,tang2018large}. In these cases, conventional raster scanning is an inefficient use of imaging time. A better alternative is holographic illumination~\cite{Yang2018Holographic,mardinly2018precise,pegard2017three,hernandez2016three,nikolenko2008slm,dal2010simultaneous,ronzitti2017recent,vaziri2012reshaping,yang2016simultaneous,dana2014hybrid,dana2012remotely}, which can specifically target neurons of interest. Ideally, the activity of these neurons can be imaged simultaneously using a fast 2D sensor. Target illumination systems for voltage imaging are becoming increasingly popular~\cite{Sims2024Scanless,Xiao2024LargeScale}, but they are limited to shallow brain layers of no more than 200-300$\mu m$  due to severe scattering aberrations. Even 2P holographic excitation, which can deliver the excitation pattern with minimal scattering, is limited by the scattering of the emitted light. Another drawback of 2P excitation is that even for shallow layers, it requires very high laser power. Thus, safety considerations restrict simultaneous excitation to no more than 10-20 neurons~\cite{Davis2024Optical}. Since voltage activity must be tracked at high kilohertz frame rates, this makes it nearly impossible to reliably monitor large neuronal populations.



Our research aims to significantly enhance the penetration depth of brain imaging and improve signal-to-noise ratios through two innovative non-invasive approaches: skull clearing techniques and wavefront shaping. Skull clearing, such as the Through-Intact-Skull (TIS) chronic window technique~\cite{Li2022TIS}, employs chemical agents that match the refractive index variations in the skull, rendering it transparent. To image through the brain tissue itself, we will integrate spatial light modulators (SLMs) into the excitation and emission paths of the microscope. These SLMs will reshape both the incoming and outgoing wavefronts, correcting aberrations caused by tissue inhomogeneity as well as remaining aberrations in the cleared skull.

In this research, we will apply wavefront shaping techniques to both 1P and 2P imaging. For 2P imaging, correcting the emitted wavefronts will eliminate the need for sequential scanning. This will allow us to use a 2D sensor to simultaneously image the voltage activity of multiple neurons, excited using a sparse hologram, at a high frame rate. Additionally, correcting the excitation wavefronts will allow us to extend 2P imaging to depths currently accessible only by less stable 3P processes\Anat{is unstable the right criticism? some other term?}. Correcting 1P wavefronts will enable this technology to reach depths comparable to those achieved with expensive 2P lasers, but at higher frame rates as we can eliminate the need for sequential scanning. Furthermore, since 1P excitation requires four orders of magnitude less power than 2P, we can simultaneously image large neuronal populations without heat concerns.

