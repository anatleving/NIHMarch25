\section*{Project Summary}

%Scattering is the hardest barrier of biomedical imaging research, and in particular for brain research. 
%Non-invasive optical imaging approach are heavily limited to superficial brain layers because light penetration is strongly aberrated as it scatters through the skull and the inhomogeneous tissue.


Optical imaging techniques hold great promise in neuroscience due to their ability to monitor neuronal activity with high spatiotemporal resolution using calcium or voltage indicators. However, these techniques encounter considerable difficulties because of light scattering. The first barrier is the skull, a highly scattering layer that impedes light penetration. Consequently, most brain research today involves highly invasive craniotomy procedures. Even after opening the skull, the effectiveness of imaging methods is largely limited to the brain's superficial layers, as light is heavily scattered through the tissue.
Brain imaging using low-power single-photon (1P) fluorescent excitation is currently confined to the first 200um of brain tissue. Expensive two-photon (2P) excitation lasers, which operate at longer wavelengths, can penetrate up to 700um before scattering becomes a problem, but they still cannot access very deep brain layers. Moreover, 2P excitation presents several significant drawbacks. Firstly, deep tissue imaging with 2P lasers requires sequential scanning, which limits the temporal resolution of recordings. Secondly, even for shallow layers, as 2P excitation demands very high laser power; safety considerations restrict simultaneous excitation to no more than 10-20 neurons. Since voltage activity must be tracked at high kilohertz frame rates, this makes it nearly impossible to reliably monitor large neuronal populations.

Our research aims at  enhancing the penetration depth of brain imaging and improving signal-to-noise ratios (SNR) through two innovative non-invasive approaches.
First, we will employ recently developed skull clearing techniques to make the skull transparent.
 Second, to image through the brain tissue itself, we will integrate spatial light modulators (SLM). These SLMs will be employed to reshape both the incoming and outgoing wavefronts, thereby correcting aberrations caused by tissue inhomogeneity as well as remaining aberrations in the cleared skull. 
We will apply wavefront shaping techniques to both 1P and 2P imaging. For 2P imaging, correcting the emitted wavefronts will eliminate the need for sequential scanning, enabling us to simultaneously excite multiple neurons using a targeted holographic illumination pattern, and simultaneously image their activity with a 2D sensor.
We further extend this idea to excite the neurons using 1P illumination with shorter wavelengths. To overcome the increased scattering of short wavelengths, we will also reshape the incoming excitation patterns.
 Since 1P excitation requires four orders of magnitude less power than 2P, we can image large neuronal populations without heat concerns.
